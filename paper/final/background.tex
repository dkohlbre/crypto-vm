\section{Background}
\label{sec:background}
CTF competitions have been including cryptography related challenges
for well over a decade. As with other challenges, the complexity of
crypto challenges has grown over time; from mostly classical ciphers
and basic operations to requiring competitors to implement cutting
edge cryptanalysis and attacks.

Most modern competitions post some version of the challenges they
developed after the competition is over. However, these challenges are
generally just raw development files with little to no explanation of
how to operate them. Many competition teams use these previous
challenges as practice, but getting them working without discovering
the solution can be impossible.

Examples of competitions that would contain relevant problems to
\cvm{} are PPP's PlaidCTF\footnote{\url{http://play.plaidctf.com/}},
LegitBS's defcon qualifiers\footnote{\url{https://legitbs.net/}},
Codegate CTF\footnote{\url{http://ctf.codegate.org/}}, and many
others. These competitions are run by both teams (PPP, LegitBS, mslc,
etc.) and companies (Codegate, PhDays). Competitions are always free
to enter, and usually open to teams world wide. While many have
prizes, competition has historically been more about bragging rights
than material gain. Notably, the (arguably) highest prestige CTF,
defcon finals, has \textit{no} monetary prize.

CTFs present an excellent source of challenges in computer science and
security for anyone interested in the field. Recently, the startup
community has begun to use CTF style competitions to look for talent
to hire\footnote{\url{http://starfighters.io/}}, and the security
industry has long used CTF performance as an indicator of
interest. Many companies and national labs also run their own versions
of CTFs internally for both training and fun.
