\section{Introduction}
In this project we present \cvm{}, a locally-hosted collection of
cryptographic puzzles. Designed to introduce users with theoretical
background to practical attacks, \cvm{} re-uses challenges from the
Capture the Flag (CTF)\cite{ctftime:whatisctf} community
competitions. \cvm{} also makes the working archival of cryptographic
CTF challenges easier, by automating the deployment and configuration
of a server to present these challenges. \cvm{} serves two userbases;
challenge authors who want to make their problems available in the
future, and participants who want to practice easily on old
challenges.


\section{Motivation}
We created \cvm{} with two primary goals in mind. First, to create an
easy to use system for people to try working on cryptographic
puzzles. Second, to preserve (in a usable format) the large number of
cryptographic challenges released by CTF competitions.

Each year, dozens of CTF competitions are held, each presenting ten to
thirty new challenges, covering topics from binary reverse engineering
and exploitation to cryptography and electrical
engineering. Unfortunately, after the competition ends, many problems
are permanently discarded or made available in unapproachable
formats. When introducing students to CTF competitions, one of the
most valuable teaching methods is to present previous CTF problems,
but with hints and support available. This way students gain the
experience of actually solving problems without the pressures of a
competition. Cryptographic challenges specifically often vary wildly
in the quality and skills needed to solve. Many of these challenges
are poorly constructed, and devolve into a guessing game of potential
cryptographic schemes or classical ciphers. This can make teaching new
competitors the skills needed to approach (for example) a challenge
consisting of a server running a incorrect textbook RSA implementation
whose secret key can be recovered extra difficult. Not only does the
mathematical background need to be shored up, but problems need to be
combed through for the right qualities, and then setup, hosted, and
tested. However, as compared to other CTF challenge types,
cryptographic challenges are especially easy to port off the
CTF-specific infrastructure. Most cryptographic challenges need only
serve static content, or a very simple (host and configuration
independent) server. \cvm{} attempts to streamline the setup and
usage, by having categorized challenges that are
pre-configured. Ideally, a student can take the background they have
(or have recently been taught) and immediately dive into relevant
challenges paired with hints and reading material.

Finally, \cvm{} attempts to make preservation of cryptographic
challenges easier, by giving challenge authors a simple and automated
system for archiving their work.

\subsection{Related Projects}
There are already some efforts within the CTF community to collect and preserve
challenges used in previous competitions. Most notably, there exists a wiki-like
Github repository \cite{ctf:github} with challenges collected from CTF
competitions since 2013. Over a hundred people have contributed to the
repository, making it the largest collection of challenges that we are aware of.

However, working with these problems is somewhat challenging as a beginner. Not
all of the problems are categorized by type. Some problems are missing
essential files, the problem descriptions are often unclear if one is not
familiar with CTF challenges, and some are lacking a proper solution writeup.
This is particularly the case for problems that run as a service that is
supposed to be contacted over the network on a specific port.

The goal of our \cvm{} is to improve this by automating such problem setup and
adding some guidance to the problems to make them more accessible. Nonetheless,
we are going to use the repository as a resource to acquire more problems for
our VM.

