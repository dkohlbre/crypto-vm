\documentclass[10pt, conference, compsocconf,hyphens]{IEEEtran}
\usepackage{mathptmx}
\usepackage{amsmath}
%\usepackage{mathtools}
\usepackage{courier}
\usepackage[scaled=.92]{helvet}

\usepackage[pdfborder={0 0 0},bookmarks=false,breaklinks,draft]{hyperref}

\usepackage{graphicx}
\usepackage{color}
\usepackage{xspace}
\usepackage{listings}
\usepackage{booktabs,tabularx}
\usepackage[numbers,sort]{natbib}
\usepackage{ragged2e}
\usepackage{tikz}
\usepackage{dcolumn}
\usepackage{dsfont}
\usepackage{pdfpages}
\usepackage{listings}

\lstset{language=C,basicstyle=\ttfamily,xleftmargin=2em}

\iffalse % Change to \iffalse for submission
\usepackage{datetime}
\usepackage{fancyhdr}
\fancypagestyle{IEEEtitlepagestyle}{%
\fancyhead{}
\fancyfoot[L]{\textcolor{red}{\bfseries DRAFT}}
\fancyfoot[C]{\thepage}
\fancyfoot[R]{\textcolor{red}{\bfseries\currenttime\ \today}}
\renewcommand\headrulewidth{0pt}
\renewcommand\footrulewidth{0pt}}
\pagestyle{IEEEtitlepagestyle}
\fi
\usepackage{flushend}

\usepackage[binary-units]{siunitx}
\usepackage{subcaption}
\usepackage{multirow}
%\usepackage[title,header]{appendix}
\usepackage{lipsum}
\usepackage[absolute]{textpos}
\usepackage[T1]{fontenc}

\providecommand{\st}{\ensuremath{\mathrm{\ s.t.\ }}}

\providecommand{\bagmap}{\texttt{bagmap}}
\providecommand{\bagsum}{\texttt{bagsum}}
\providecommand{\bagsize}{\texttt{bagsize}}
\providecommand{\bagsplit}{\texttt{bagsplit}}

\providecommand{\lone}{\ensuremath{\ell{}_1}}

\hyphenation{white-list}

% taken from hs's mary.tex, and tracing back to knuth.
\def\dash---{\kern.16667em---\penalty\exhyphenpenalty\hskip.16667em\relax}

% C++ macro from john mitchell
\def\CC{C\raise.22ex\hbox{{\footnotesize +}}\raise.22ex\hbox{\footnotesize +}\xspace}

% Make URLs linebreak better (hat-tip alexras)
  % A sequence of BigBreaks will be treated as one break, so it will only be able to break after ://
  \renewcommand{\UrlBigBreaks}{\do\:\do\/}
  % (Less aggressive) Treat both / and - as breakable characters (don't know why this does something different than hyphens in the package declaration, but it does)
  \renewcommand{\UrlBreaks}{\do\/\do\-}
  % (More aggressive) Any letter and / are treated as breakable characters
  \renewcommand{\UrlBreaks}{\do\/\do\a\do\b\do\c\do\d\do\e\do\f\do\g\do\h\do\i\do\j\do\k\do\l\do\m\do\n\do\o\do\p\do\q\do\r\do\s\do\t\do\u\do\v\do\w\do\x\do\y\do\z\do\A\do\B\do\C\do\D\do\E\do\F\do\G\do\H\do\I\do\J\do\K\do\L\do\M\do\N\do\O\do\P\do\Q\do\R\do\S\do\T\do\U\do\V\do\W\do\X\do\Y\do\Z}

\lstdefinelanguage{JavaScript}{
  morekeywords={typeof, new, true, false, catch, function, return, null, catch, switch, var, if, in, while, do, else, case, break},
  morecomment=[s]{/*}{*/},
  morecomment=[l]//,
  morestring=[b]'',
  morestring=[b]'
}

\lstdefinelanguage{HTML5}{
        language=html,
        sensitive=true,
        alsoletter={<>=-},
        otherkeywords={
        % HTML tags
        <feConvolveMatrix, />
        },
        ndkeywords={
        % General
        =,
        % HTML attributes
        charset=, id=, width=, height=, in=, order=, edgeMode=, kernelMatrix=, preserveAlpha=,
        % CSS properties
        border:, transform:, -moz-transform:, transition-duration:, transition-property:, transition-timing-function:,
        },
        morecomment=[s]{<!--}{-->},
        tag=[s]
}

\lstset{%
    % Code
    language=HTML5,
    alsolanguage=JavaScript,
    showstringspaces=false,
    extendedchars=true,
    breaklines=true
 }

\providecommand{\setjmp}{\texttt{setjmp}}
\providecommand{\longjmp}{\texttt{longjmp}}

% Alter some LaTeX defaults for better treatment of figures:
    % See p.105 of "TeX Unbound" for suggested values.
    % See pp. 199-200 of Lamport's "LaTeX" book for details.
    %   General parameters, for ALL pages:
    \renewcommand{\topfraction}{0.99}    % max fraction of floats at top
    \renewcommand{\bottomfraction}{0.8} % max fraction of floats at bottom
    %   Parameters for TEXT pages (not float pages):
    \setcounter{topnumber}{2}
    \setcounter{bottomnumber}{4}
    \setcounter{totalnumber}{6}         % 2 may work better
    \setcounter{dbltopnumber}{6}        % for 2-column pages
    \renewcommand{\dbltopfraction}{0.99} % fit big float above 2-col. text
    \renewcommand{\textfraction}{0.07}  % allow minimal text w. figs
    %   Parameters for FLOAT pages (not text pages):
    \renewcommand{\floatpagefraction}{0.7}  % require fuller float pages
    % N.B.: floatpagefraction MUST be less than topfraction !!
    \renewcommand{\dblfloatpagefraction}{0.7} % require fuller float pages


\setlength{\TPHorizModule}{30mm}
\setlength{\TPVertModule}{\TPHorizModule}
\textblockorigin{10mm}{10mm}

\newcolumntype{C}[1]{>{\centering\let\newline\\\arraybackslash\hspace{0pt}}m{#1}}

% get rid of that stupid box on the first page.
\makeatletter
\def\@copyrightspace{}
\makeatother

% Support cool highlighting boxes in lstlisting (from
% http://tex.stackexchange.com/questions/15237/highlight-text-in-code-listing-while-also-keeping-syntax-highlighting)
\makeatletter
\newenvironment{btHighlight}[1][]
{\begingroup\tikzset{bt@Highlight@par/.style={#1}}\begin{lrbox}{\@tempboxa}}
{\end{lrbox}\bt@HL@box[bt@Highlight@par]{\@tempboxa}\endgroup}

\newcommand\todo[1]{\textcolor{red}{TODO:#1}}

\newcommand\btHL[1][]{%
  \begin{btHighlight}[#1]\bgroup\aftergroup\bt@HL@endenv%
}
\def\bt@HL@endenv{%
  \end{btHighlight}%
  \egroup
}
\newcommand{\bt@HL@box}[2][]{%
  \tikz[#1]{%
    \pgfpathrectangle{\pgfpoint{1pt}{0pt}}{\pgfpoint{\wd #2}{\ht #2}}%
    \pgfusepath{use as bounding box}%
    \node[anchor=base west, fill=orange!30,outer sep=0pt,inner xsep=1pt, inner ysep=0pt, rounded corners=3pt, minimum height=\ht\strutbox+1pt,#1]{\raisebox{1pt}{\strut}\strut\usebox{#2}};
  }%
}
\lstdefinestyle{Chighlight}{
    language={C},basicstyle=\ttfamily,
    moredelim=**[is][\btHL]{`}{`},
    moredelim=**[is][{\btHL[fill=green!30,draw=red,dashed,thin]}]{@}{@},
}
\makeatother


% support aligning a table column on .
\newcolumntype{d}[1]{D{.}{.}{#1} }

\def\sharedaffiliation{
\end{tabular}
\begin{tabular}{c}}
\begin{document}


\title{Crypto-vm: A self hosted introduction to cryptography puzzles}

\author{
\IEEEauthorblockN{David Kohlbrenner, Benjamin Braun}
\IEEEauthorblockA{
University of California, San Diego\\
}
}


\clubpenalty=10000
\widowpenalty=10000

\maketitle

\section{Introduction}
In this project we present \cvm{}, a locally-hosted collection of
cryptographic puzzles. Designed to introduce users with theoretical
background to practical attacks, \cvm{} re-uses challenges from the
Capture the Flag (CTF)\cite{ctftime:whatisctf} community
competitions. \cvm{} also makes the working archival of cryptographic
CTF challenges easier, by automating the deployment and configuration
of a server to present these challenges. \cvm{} serves two userbases;
challenge authors who want to make their problems available in the
future, and participants who want to practice easily on old
challenges.


\section{Motivation}
We created \cvm{} with two primary goals in mind. First, to create an
easy to use system for people to try working on cryptographic
puzzles. Second, to preserve (in a usable format) the large number of
cryptographic challenges released by CTF competitions.

Each year, dozens of CTF competitions are held, each presenting ten to
thirty new challenges, covering topics from binary reverse engineering
and exploitation to cryptography and electrical
engineering. Unfortunately, after the competition ends, many problems
are permanently discarded or made available in unapproachable
formats. When introducing students to CTF competitions, one of the
most valuable teaching methods is to present previous CTF problems,
but with hints and support available. This way students gain the
experience of actually solving problems without the pressures of a
competition. Cryptographic challenges specifically often vary wildly
in the quality and skills needed to solve. Many of these challenges
are poorly constructed, and devolve into a guessing game of potential
cryptographic schemes or classical ciphers. This can make teaching new
competitors the skills needed to approach (for example) a challenge
consisting of a server running a incorrect textbook RSA implementation
whose secret key can be recovered extra difficult. Not only does the
mathematical background need to be shored up, but problems need to be
combed through for the right qualities, and then setup, hosted, and
tested. However, as compared to other CTF challenge types,
cryptographic challenges are especially easy to port off the
CTF-specific infrastructure. Most cryptographic challenges need only
serve static content, or a very simple (host and configuration
independent) server. \cvm{} attempts to streamline the setup and
usage, by having categorized challenges that are
pre-configured. Ideally, a student can take the background they have
(or have recently been taught) and immediately dive into relevent
challenges paired with hints and reading material.

Finally, \cvm{} attempts to make preservation of cryptographic
challenges easier, by giving challenge authors a simple and automated
system for archiving their work.

\section{Background}
\label{sec:background}
TODO:david

\section{Methodology}
\label{sec:methodology}
We used a number of tools to make \cvm{} as portable and accessible as possible.

To avoid having to distribute a large virtual machine image file, we decided to use Vagrant to build our VM. Vagrant builds and configures a virtual machine from a small configuration file. We use an Ubuntu 14.04 base image that is automatically downloaded when building \cvm{} for the first time. We have developed a number of Python and Bash scripts that will then generate HTML pages for the available CTF problems, configure and run services for problems that require them and start a webserver on the VM. Not considering the time required to download the base image, it only takes a few seconds until the VM is fully provisioned and ready to be used. 

\subsection{Problem structure}
Each CTF problem has a strictly defined structure to allow for automated page generation. Each problem is contained in a directory and should at least include the following items:

\begin{itemize}
  \item A setup script to register and start any necessary services and copy required public files, for example source code or ciphertexts, to the webserver directory. 
  \item A description markdown file that is similar to a typical CTF problem description, but a little more detailed and helpful, to make the problem more accessible to our target audience.
  \item A hints markdown file that includes helpful pointer in the right direction to help beginners. It should also include a section with recommended background reading material that help familiarize with the type of problem without directly giving the answer. If applicable, this can also include references to tools or libraries that can be helpful for solving the problem.
  \item A solution markdown file. Preferably, this should include an easy to follow explenation of a solution to the problem. It is also acceptable to link to an external writeup of the problem, but then a cached PDF version of the writeup needs to be included to make sure that a solution is always accessible.
  \item A file containing the flag that will be matched on the page.
  \item A configuration file describing the category of the problem and other configuration values.
  \item If needed, additional file required for solving the problem, for example source code, ciphertexts, or xinetd configuration files.
\end{itemize}

To make it easier to add new problems, we have created a setup script that guides through the process of adding a new problem to the VM by generating a template with all the required files.


\section{Problem Examples}
\subsection{Hitcon 2014's rsaha}
todo:ben
\subsection{PlaidCTF 2014's parlor}
todo:david

\section{Future Work}

- add more problems
- improve design
- problem randomization -> n random problems so people don't get overwellmed
    competition mode without hints and description

    - publicing, get some pr so that other start using it, add problems to it 
    - add difficulty ratings, sort by them 

\section{Conclude}
its rad


{\small
\bibliographystyle{IEEEtranSN}
\bibliography{final}
}

\typeout{}

\end{document}
